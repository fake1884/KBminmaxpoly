
\newglossaryentry{Y}
{
    name=Y,
    description={ $=(y_1, \ldots, y_n)$ Daten, abhängige Parameter}
}
 
\newglossaryentry{X}
{
    name=X,
    description={$=(x_1, \ldots, x_p)=\begin{bmatrix} x_{1,1} & \ldots & x_{1,n} \\ \vdots & \ddots & \vdots \\ x_{p,1} & \ldots & x_{p,n} \end{bmatrix}$ Designmatrix, unabhängiger Parameter}
}

\newglossaryentry{P(...)}
{
    name=$\mathbb{P}(...)$,
    description={ Wahrscheinlichkeitsmaß }
} 

\newglossaryentry{e}
{
    name=$e$,
    description={ Fehler bei der Regression. $\hat{e}$ ist immer ein Schätzer für e }
} 
 
%\newglossaryentry{e-hat}
%{
%    name=$\hat{e}$,
%    description={ Schätzer für den Fehler}
%} 

\newglossaryentry{sigma-square}
{
    name=$\sigma^2$,
    description={ Varianz der Fehler. $\widehat{\sigma^2}$ ist immer ein Schätzer für $\sigma^2$ }
} 

\newglossaryentry{beta}
{
    name=$\beta$,
    description={$=(\beta_1, \ldots, \beta_p)$ Koeffizientenvektor, unbekannte Konstante. $\hat{\beta}$ ist immer ein Schätzer für $\beta$}
} 

\newglossaryentry{A}
{
    name=A,
    description={$= \{(x_1, \ldots, x_p)' : a_i \leq x_i \leq b_i, i = 1, \ldots, p \} \subset \mathbb{R}^n$ für $- \infty \leq a_i < b_i \leq \infty, i=1, \ldots, p$ Gebiet das einen interessiert. Dabei steht A für Area}
} 
 

\newglossaryentry{p}
{
    name=p,
    description={ Anzahl an unabhängigen Parametern}
} 

\newglossaryentry{n}
{
    name=n,
    description={ Anzahl an abhängigen Parametern}
} 

\newglossaryentry{v}
{
    name=v,
    description={ = n-p-1}
} 

\newglossaryentry{x-prime}
{
    name=$x'$,
    description={ Transponierte von x $\in \mathbb{R}^n$}
} 

\newglossaryentry{Kovarianz}
{
    name=$\text{Cov}(u;v)$,
    description={ Kovarianz zwischen $u$ und $v$}
} 

\newglossaryentry{Erwartungswert}
{
    name=E(W),
    description={ Erwartungswert von W}
}  
 
\newglossaryentry{kritischer-Wert}
{
    name=c,
    description={ kritischer Wert bei der Berechnung von Konfidenzbändern}
} 

%\newglossaryentry{Polynomgrad}
%{
%    name=k,
%    description={ Polynomgrad}
%} 

\newglossaryentry{Wkeit}
{
    name=$\alpha$,
    description={ Konfidenzwahrscheinlichkeit für Tests und Konfidenzbänder}
}  
 
\newglossaryentry{S}
{
    name=S,
    description={ =$\sup_{x\in A} \frac{\vert X'(\beta-\hat{\beta}) \vert}{\hat{\sigma} \sqrt{x'(X'X)^{-1}x}}$ ist ein Bruch von dem oft die Verteilung interessiert. Siehe auch Satz \ref{KB_Eigenschaft}}
}  

\newglossaryentry{Normalverteilung}
{
    name=$\mathscr{N}_{p+1}(E;V)$,
    description={ ist die (p+1)-dimensionale Normalverteilung mit Erwartungswert E und Varianz V. Bei der eindimensionalen Normalverteilung entfällt der Index}
} 

\newglossaryentry{X-invers}
{
    name=$X^{-1}$,
    description={ inverse von X}
} 
 
\newglossaryentry{chi-square}
{
    name=$\chi_v^2$,
    description={ Chi-Quadrat Verteilung mit $v$ Freiheitsgraden}
}  
 
\newglossaryentry{Matrizen}
{
    name=Matrizen,
    description={ Große lateinische Buchstaben wie Q,T,L sind Matrizen mit bekannten Werten. Große griechische Buchstaben wie $\Upsilon$ sind für Matrizen mit unbekannten Werten}
}  
 
\newglossaryentry{P}
{
    name=P,
    description={ ist die Wurzel aus $(X'X)^{-1}$. Also ist $P^2=(X'X)^{-1}$}
}  

\newglossaryentry{N}
{
    name=N,
    description={ $=P^{-1}(\hat{\beta}-\beta)/\sigma$ wird manchmal in Beweisen benutzt}
} 
 
\newglossaryentry{f}
{
    name=$f^{\alpha}_{p+1;n-p-1}$,
    description={ $\alpha$-Quantil der F-Verteilung mit den Parametern $p+1$ und $n-p-1$}
} 

\newglossaryentry{phi}
{
    name=$\phi$,
    description={ Parameter bei AR(1)-Prozessen}
} 

\newglossaryentry{I}
{
    name=I,
    description={ ist die Einheitsmatrix mit passender Dimension}
} 

\newglossaryentry{D}
{
    name=D,
    description={ $=(X_1'X_1)^{-1}+(X_2'X_2)^{-1}$}
} 

\newglossaryentry{T}
{
    name=T,
    description={ in Abschnitt \ref{Konfidenzbaender vergleich} ist $T=P^{-1}(\hat{\beta_2}-\beta_2-\hat{\beta_1}+\beta_1)/\hat{\sigma}$. In Abschnitt \ref{Regression für AR(1)} steht T für die Anzahl an Elementen einer Zeitreihe}
} 

%\newglossaryentry{phi}
%{
%    name=$\phi$,
%    description={ Parameter bei AR(1)-Prozessen}
%} 

\newglossaryentry{M}
{
    name=M,
    description={ Menge werden normalerweise mit M bezeichnet}
} 

\newglossaryentry{Tau}
{
    name=$\tau_{p+1,v}$,
    description={ ist die $\tau$-Verteilung mit Parametern $p+1$ und $v$. Sie taucht im Abschnitt \ref{Konfidenzbaender auf R fuer ein multiples lineares Regressionsmodell} auf und wird dort auch definiert}
} 

\newglossaryentry{C}
{
    name=C(P;A),
    description={ Gebiet in Abschnitt \ref{Konfidenzbaender auf einem Intervall fuer ein multiples lineares Regressionsmodell} mit P und A wie im Variablenverzeichnis}
}

\newglossaryentry{Pi}
{
    name=$\pi(t;P;A)$,
    description={ ist die Projektion von t auf C(P,A)}
}

\newglossaryentry{X-Tilde}
{
    name=$\tilde{x}$,
    description={ $=(1, x, x^2, \ldots, x^p) \subset \mathbb{R}^{p+1}$ ist ein konkreter Wert des abhängigen Parameters, falls der abhängige Parameter Polynomgestalt hat}
}

\newglossaryentry{K}
{
    name=$K_{2h}(T;(X'X)^{-1};A)$,
    description={ ist eine wichtige Funktion, die in Abschnitt \ref{Konfidenzbaender auf einem Intervall fuer ein multiples lineares Regressionsmodell} definiert wird}
}

\newglossaryentry{H}
{
    name=H,
    description={ $=X(X'X)^{-1}X'$ taucht im ersten Kapitel auf}
}  

\newglossaryentry{Nullhypothese}
{
    name=Nullhypothese,
    description={ $H_0$ und $H_1$ sind die Null-und Alternativhypothese beim Hypothesentest}
}

\newglossaryentry{Xi}
{
    name=$\Upsilon$,
    description={ Varianzmatrix bei AR(1)-Prozessen}
}

%\newglossaryentry{delta}
%{
%    name=$\delta$,
%    description={ taucht in Satz 2.1.2 auf}
%}

%\newglossaryentry{z}
%{
%    name=z,
%    description={ ist eine Dummyvariable, die in Abschnitt \ref{Vergleich Konfidenzbaender auf ganz R} auftaucht.}
%}

%\newglossaryentry{c}
%{
%    name=c,
%    description={ $c_1$ und $c_2$ tauchen in Abschnitt \ref{Vergleich F-Test} auf.}
%}